\documentclass[a4paper,twoside,10pt]{article}

\usepackage[francais]{babel}
\usepackage[T1]{fontenc}
\usepackage[utf8]{inputenc}

\usepackage{graphicx} 
\usepackage{qtree} %arbres
\renewcommand{\H}{\mathcal{H}}
\newcommand{\I}{\mathcal{I}}
\newcommand{\B}{\mathcal{B}}
\newcommand{\D}{\mathcal{D}}
\usepackage{amssymb}
\usepackage{amsmath}
\usepackage{amsfonts}
%\usepackage{amsthm}

\usepackage{setspace} %\onehalfspacing       %% 1,5-spacing
%\usepackage{a4wide} %%Smaller margins = more text per page.

%\usepackage{minted}
%\usemintedstyle{colorful}
\newenvironment{Q}[1]{%
\vspace{1ex}
\underline{\textbf{Question #1\\}}
\newline
}{
\vspace{2ex}
}

\title{Entropie et codage de source}
\author{Alice Andrès, Quentin SOubeyran}

\begin{onehalfspacing}
\begin{document}
\maketitle

\section{Entropie d'une distribution de probabilité}

\subsection{Cadre de travail et idée intuitive}
\begin{Q}{1}

\end{Q}
\subsection{Entropie relative et information mutuelle}
\begin{Q}{2}
$\D(\B(a)||\B(b))
= a log_2(\frac{a}{b}) + (1-a) log_2(\frac{1-a}{1-b})$

D'où $\D(\B(a)||\B(b))  - \D(\B(b)||\B(a)) 
= (a+b)\log_2{\frac{a}{b}} + (2-a-b)\log_2{\frac{1-a}{1-b}}$

Or pour $a = \frac{1}{4}$ et $b = \frac{1}{2}$, 
\[
\D(\B(a)||\B(b))  - \D(\B(b)||\B(a))
= \frac{3}{4} \log_2{(2)} + \frac{5}{4} \log_2{(\frac{3}{2})} \neq 0
\]


Ainsi, dans le cas général, $ \D (p||q) \neq \D(q||p)$
\end{Q}

\begin{Q}{3a}
La fonction $- \log_2$ est strictement convexe. Alors, d'après l'inégalité de Jensen, 

$\sum_{x \in E} p(x) (-\log_2{(\frac{q(x)}{p(x)})}) \geq 
-\log_2{(\sum_{x \in E}p(x) \frac{q(x)}{p(x)})} = -log_2{(\sum_{x \in E}q(x))} = 0$

Ainsi $D(p||q) \geq 0$

La stricte convexité de $-\log_2$ permet de conclure qu'il y a égalité si et seulement si $\forall x \in E$, $p(x) = q(x)$, soit $p=q$. 
\end{Q}

\begin{Q}{3b}

D'après Q3a, $\I(X,Y) = \D(p_{(X,Y)} || p_X \otimes p_Y) \geq 0$

Avec égalité si et seulement si $p_{(X,Y)} = p_X \otimes p_Y \iff X$ et $Y$ sont indépendants.
\end{Q}

\begin{Q}{4a}

$\H(X,Y) = - \sum_{x,y \in E²}  p_{X,Y}(x,y)\log_2{(p_{X,Y}(x,y))}\\ = 
-\sum_{x \in E} \sum_{y \in E} p_X(x) p_{Y|X = x}(y) \log_2{p_X(x)} + \log_2{p_{Y|X = x}(y)}\\ =
\H(X) + \sum_{x \in E} p_X(x) (-\sum_{Y \in E} p_{Y|X = x}(y) \log_2{p_{Y|X = x}(y)})\\
 = \H(X) + \H(Y|X)$
\end{Q} 
 
\begin{Q}{4b}
 
$\I(X,Y) = \sum_{(X,Y) \in E²} p_{X,Y}(x,y) \log_2{\frac{p_{X,Y}{x,y}}{p_X(x)p_Y(y)}} \\
 = \sum_{(X,Y) \in E²} p_X(x) p_{Y|X=x}(y) \log_2{(p_{Y|X=x}(y))} 
 - \sum_{(X,Y) \in E²} p_Y(y) p_{X|Y=y}(x) \log_2{(p_{Y}(y))}\\
 = \H(Y) - \H(Y|X)\\
= \H(X) - \H(X|Y)$ (par symétrie des rôles de X et Y) 

$= \H(Y) - (\H(X,Y) - \H(X))$ (Q4a)

$= \H(X) + \H(Y) - \H(X,Y)$
\end{Q}

\begin{Q}{4c}

D'après Q4b, $\H(X,Y) = \H(X) - \I(X;Y)$ Or $\I(X;Y) \geq 0$

Ainsi, $\H(X,Y) \leq \H(X)$
\end{Q}

\begin{Q}{5a}

On utilise l'algorithme d'inversion de la fonction de répartition pour une loi discrète.

On utilise python pour déterminer un nombre $a$ aléatoirement suivant la loi uniforme, entre 0 et 1, et on pose Y tel que : 

$Y = x_i  \iff \sum_{j = 1}^{i-1} p_j < a \leq \sum_{j = 1}^{i} p_{j+1}$

On peut appliquer ce principe pour $X \leadsto \B(\frac{1}{3})$

Soit $a \leadsto \mathcal{U}([0;1])$ Notons aussi $x_0 = 1$ et $x_1 = 0$

Alors $\mathbb{P}(X = x_0) = \frac{2}{3} = \mathbb{P}(a < \frac{2}{3})$ et 
$\mathbb{P}(X = x_1) = \frac{1}{3} = \mathbb{P}(a > \frac{2}{3})$.
\end{Q}

\begin{Q}{5b}

\end{Q}

\begin{Q}{6a}

\end{Q}

\begin{Q}{6b}
\end{Q}

\begin{Q}{6c}
\end{Q}

\section{Application au codage de source}
\subsection{Théorème du codage de source}

\begin{Q}{7a}

\begin{align*}
\D(p_X||q) &= \sum_{x \in E}p_X(x)log_2(\frac{p_X(x)}{\frac{1}{c}d^{-l(x)}}) \\
&\geq 0
\end{align*}

Alors
\begin{align*}
\sum_{x \in E}p_X(x)log_2(p_X(x)) &\geq  -\sum_{x \in E}p_X(x)l(x)log_2(d) + \sum_{x \in E}p_X(x)log_2(\frac{1}{c}) \\
 &\iff\\
-\H(X) &\geq -log_2(d) \mathbb{E}(X) + log_2(\frac{1}{c})
\\ &\geq -log_2(d) \mathbb{E}(X) \texttt{\quad(car} \leq 1 \textsc{)}
\end{align*}

D'où $\frac{\H(x)}{log_2(d)} \leq \mathbb{E}[l(X)]$
 
Le cas d'égalité se déduit de celui de $\D$, et a lieu pour $p_X = q$, soit les $p_X(x)$ sont des puissances négatives de $d$.
\end{Q}

\begin{Q}{7b}
Soit $p$ une loi de probabilité telle que qui s'écrit $p_X(x) = \frac{1}{c}d^{-n_x}$ avec $c = \sum_{x \in E} d^{-n_x}$.

Cas 1 : $c \leq 1$ Prenons $\forall x \in E, l_0(x) = n_x$

Cas 2 : $c > 1$ Alors soit $k$ tel que $\frac{c}{d^k} \leq 1$

$p_X(x) = \frac{d^k}{c}d^{-n_x - k}$, avec $\sum_{x \in E} d^{-n_x - k} \leq \frac{c}{d^k} \leq 1$

Posons alors $\forall x \in E, l_0(x) = n_x + k$

Cette application vérifie l'inégalité de Kraft-McMillan, et vérifie le cas d'égalité de la question Q7a d'après les calculs
précédents pour $q$ définie à partir de la fonction $l_0$.
\end{Q}

\begin{Q}{7c}
La fonction puissance étant bijective sur $\mathbb{R^+}$, on a :
\[
\forall x \in E, \exists \alpha_x, \quad p_X(x) = d ^{\alpha_x}
\]
Posons $c$ et $\beta$ tels que :
\[
c = \sum_{x \in E} d ^{\alpha_x} = d^\beta
\]

Alors
\[
\forall x \in E, \quad p_X(x) = \frac{1}{c} d^{-(\beta - \alpha_x)}
\]
On pose donc
\[
l_0(x) = \beta - \alpha_x
\]
D'où
\begin{align*}
\mathbb{E}[\overline{l_0}(X)] &= \sum_{x \in E} \overline{l_0}(X) \mathbb{P}(X=x)\\
&< \sum_{x \in E} l_0(X)\mathbb{P}(X=x) + \sum_{x \in E}\mathbb{P}(X=x)
\end{align*}
Or d'après la question Q7a, la forme de $p_X(x)= \frac{1}{c} d^{-(\beta - \alpha_x)}$ assure :
\[
\frac{\H(x)}{log_2(d)} = \mathbb{E}[l(X)] \text{\quad puisque } \D(p_X||p_X) = 0
\]
\end{Q}

\subsection{Mise en œuvre de l'algorithme - L'algorithme de Huffman}

\begin{Q}{9a}
Voici le tableau des occurrences.

\begin{tabular}{|c|c|c|c|c|c|}
\hline
a & b & c & d & e & f \\
\hline
2 & 3 & 1 & 2 & 2 & 1 \\
\hline
\end{tabular}

On choisit c et f

\begin{tabular}{|c|c|c|c|c|}
\hline
a & b & d & e & cf \\
\hline
2 & 3 & 2 & 2 & 2 \\
\hline
\end{tabular}

On choisit e et cf

\begin{tabular}{|c|c|c|c|}
\hline
a & b & d & ecf \\
\hline
2 & 3 & 2 & 4 \\
\hline
\end{tabular}

On choisit a et d

\begin{tabular}{|c|c|c|}
\hline
b & ad & ecf \\
\hline
3 & 4 & 4 \\
\hline
\end{tabular}

On choisit b et ad

\begin{tabular}{|c|c|c|}
\hline
bad & ecf \\
\hline
7 & 4 \\
\hline
\end{tabular}

On n'a plus que deux éléments, et construisons donc l'arbre en remontant les étapes précédentes.

\Tree [. bad ecf ]

On décompose bad en b et ad

\Tree [. [.bad b ad ] ecf ]

On décompose ad en a et d

\Tree [. [.bad b [.ad a d ] ] ecf ]

On décompose ecf en e et cf

\Tree [. [.bad b [.ad a d ] ] [.ecf e cf ] ]

On décompose cf en c et f

\Tree [. [.bad b [.ad a d ] ] [.ecf e [.cf c f ] ] ]

On en déduit le codage de Huffman : 

\begin{tabular}{|c|c|c|c|c|c|}
\hline
a & b & c & d & e & f \\
\hline
010 & 00 & 110 & 011 & 10 & 111 \\
\hline
\end{tabular}
\end{Q}

%\inputminted[linenos=true, fontsize=\small]{python}{dijkstra5.py}
\end{onehalfspacing}
\end{document}
