\documentclass[a4paper,twoside,10pt]{article}

\usepackage[francais]{babel}
\usepackage[T1]{fontenc}
\usepackage[utf8]{inputenc}
\usepackage{gfsartemisia}

\usepackage{graphicx} 
\renewcommand\H{\mathcal{H}}
\renewcommand\I{\mathcal{I}}
\renewcommand\B{\mathcal{B}}
\renewcommand\D{\mathcal{D}}
%\usepackage{amssymb}
%\usepackage{amsmath}
\usepackage{amsfonts}
%\usepackage{amsthm}
%\usepackage{stmaryrd}%intervalles d'entiers

\usepackage{setspace} %\onehalfspacing       %% 1,5-spacing
\usepackage{a4wide} %%Smaller margins = more text per page.

\usepackage{minted}
%\usemintedstyle{colorful}

%\usepackage[top=1,5cm, bottom=1,5cm, left=1cm, right=1cm]{geometry}

\begin{onehalfspacing}
\begin{document}

\subsubsection*{Q2}
$\D(\B(a)||\B(b))
= a log_2(\frac{a}{b}) + (1-a) log_2(\frac{1-a}{1-b})$

D'où $\D(\B(a)||\B(b))  - \D(\B(b)||\B(a)) 
= (a+b)\log_2{\frac{a}{b}} + (2-a-b)\log_2{\frac{1-a}{1-b}$

Or pour $a = \frac{1}{4}$ et $b = \frac{1}{2}$, 
$\D(\B(a)||\B(b))  - \D(\B(b)||\B(a))
= \frac{3}{4} \log_2{(2)} + \frac{5}{4} \log_2{(\frac{3}{2})} \neq 0 $

Ainsi, dans le cas général, $\D(p||q) \nequal \D(q||p)$

\subsubsection*{Q3a}
La fonction $- \log_2$ est convexe. Alors, d'après l'inégalité de Jensen, 

$\sum_{x \in E} p(x) (-\log_2{(\frac{q(x)}{p(x)})}) \geq 
-\log_2{(\sum_{x \in E}p(x) \frac{q(x)}{p(x)})} = -log_2{(\sum_{x \in E}q(x))} = 0$

Ainsi $D(p||q) \geq 0$

Si 

\subsubsection*{Q3b}

D'après Q3a, $\I(X,Y) = \D(p_{(X,Y)} || p_X \otimes p_Y) \geq 0$

Avec égalité si et seulement si $p_{(X,Y)} = p_X \otimes p_Y \iff X$ et $Y$ sont indépendants.

\subsubsection*{Q4a}

$\H(X,Y) = - \sum_{x,y \in E²}  p_{X,Y}(x,y)\log_2{(p_{X,Y}(x,y))}\\ = 
-\sum_{x \in E} \sum_{y \in E} p_X(x) p_{Y|X = x}(y) \log_2{p_X(x)} + \log_2 p_{Y|X = x}(y)}\\ =
\H(X) + \sum_{x \in E} p_X(x) (-\sum_{Y \in E} p_{Y|X = x}(y) \log_2 p_{Y|X = x}(y)})\\
 = \H(X) + \H(Y|X)$
 
 \subsubsection*{Q4b}
 
$\I(X,Y) = \sum_{(X,Y) \in E²} p_{X,Y}(x,y) \log_2{\frac{p_{X,Y}{x,y}}{p_X(x)p_Y(y)}} \\
 = \sum_{(X,Y) \in E²} p_X(x) p_{Y|X=x}(y) \log_2{(p_{Y|X=x}(y))} 
 - \sum_{(X,Y) \in E²} p_Y(y) p_{X|Y=y}(x) \log_2{(p_{Y}(y))}\\
 = \H(Y) - \H(Y|X)\\
= \H(X) - \H(X|Y)$ (par symétrie des rôles de X et Y) 

$= \H(Y) - (\H(X,Y) - \H(X))$ (Q4a)

$= \H(X) + \H(Y) - \H(X,Y)$


\subsubsection*{Q4c}

D'après 4b, $\H(X,Y) = \H(X) - \I(X;Y)$ Or $\I(X;Y) \geq 0$

Ainsi, $\H(X,Y) \leq \H(X)$

\subsubsection*{Q5a}

On utilise l'algorithme d'inversion de la fonction de répartition pour une loi discrète.

On utilise python pour déterminer un nombre $a$ aléatoirement suivant la loi uniforme, entre 0 et 1, et on pose Y tel que : 

$Y = x_i  \iff \sum_{j = 1}^{i-1} p_j < a \leq \sum_{j = 1}^{i} p_{j+1}$

On peut appliquer ce principe pour $X \leadsto \B(\frac{1}{3})$

Soit $a \leadsto \mathcal{U}([0;1])$ Notons aussi $x_0 = 1$ et $x_1 = 0$

Alors $\mathbb{P}(X = x_0) = \frac{2}{3} = \mathbb{P}(a < \frac{2}{3})$ et 
$\mathbb{P}(X = x_1) = \frac{1}{3} = \mathbb{P}(a > \frac{2}{3})$.

\subsubsection*{Q7a}

$\D(p_X||q) = \sum_{x \in E}p_X(x)log_2(\frac{p_X(x)}{\frac{1}{c}d^{-l(x)}}) \geq 0$

Alors $\sum_{x \in E}p_X(x)log_2(p_X(x)) \geq  -\sum_{x \in E}p_X(x)l(x)log_2(d) + \sum_{x \in E}p_X(x)log_2(\frac{1}{c}) \\
 \iff -\H(X) \geq -log_2(d) \mathbb{E}(X) + log_2(\frac{1}{c}) \geq -log_2(d) \mathbb{E}(X)$ (car $c \leq 1$)
 D'où $\frac{\H(x)}{log_2(d)} \leq \mathbb{E}[l(X)]$
 
 Le cas d'égalité se déduit de celui de $\D$, et a lieu pour $p_X = q$, soit les $p_X(x)$ sont des puissances négatives de $d$
 Le théorème 2 permert de dire que si $p_X$ est défini comme des puissances négatives de $d$, 
 
 \n 
 
 ??? Et pour c = 1?????
 
 \subsubsection*{Q7b}
 


%\subsection{Code}
%\subsubsection*{Automate}
%\inputminted[linenos=true, fontsize=\small]{python}{AutomateTourner 6.4.2.py}
%\subsubsection*{Dijkstra et Graphe}
%\inputminted[linenos=true, fontsize=\small]{python}{dijkstra5.py}

\end{onehalfspacing}
\end{document}




